\documentclass[]{article}
\usepackage{lmodern}
\usepackage{amssymb,amsmath}
\usepackage{ifxetex,ifluatex}
\usepackage{fixltx2e} % provides \textsubscript
\ifnum 0\ifxetex 1\fi\ifluatex 1\fi=0 % if pdftex
  \usepackage[T1]{fontenc}
  \usepackage[utf8]{inputenc}
\else % if luatex or xelatex
  \ifxetex
    \usepackage{mathspec}
  \else
    \usepackage{fontspec}
  \fi
  \defaultfontfeatures{Ligatures=TeX,Scale=MatchLowercase}
\fi
% use upquote if available, for straight quotes in verbatim environments
\IfFileExists{upquote.sty}{\usepackage{upquote}}{}
% use microtype if available
\IfFileExists{microtype.sty}{%
\usepackage{microtype}
\UseMicrotypeSet[protrusion]{basicmath} % disable protrusion for tt fonts
}{}
\usepackage[margin=1in]{geometry}
\usepackage{hyperref}
\hypersetup{unicode=true,
            pdftitle={Charles Clarke},
            pdfborder={0 0 0},
            breaklinks=true}
\urlstyle{same}  % don't use monospace font for urls
\usepackage{color}
\usepackage{fancyvrb}
\newcommand{\VerbBar}{|}
\newcommand{\VERB}{\Verb[commandchars=\\\{\}]}
\DefineVerbatimEnvironment{Highlighting}{Verbatim}{commandchars=\\\{\}}
% Add ',fontsize=\small' for more characters per line
\usepackage{framed}
\definecolor{shadecolor}{RGB}{248,248,248}
\newenvironment{Shaded}{\begin{snugshade}}{\end{snugshade}}
\newcommand{\AlertTok}[1]{\textcolor[rgb]{0.94,0.16,0.16}{#1}}
\newcommand{\AnnotationTok}[1]{\textcolor[rgb]{0.56,0.35,0.01}{\textbf{\textit{#1}}}}
\newcommand{\AttributeTok}[1]{\textcolor[rgb]{0.77,0.63,0.00}{#1}}
\newcommand{\BaseNTok}[1]{\textcolor[rgb]{0.00,0.00,0.81}{#1}}
\newcommand{\BuiltInTok}[1]{#1}
\newcommand{\CharTok}[1]{\textcolor[rgb]{0.31,0.60,0.02}{#1}}
\newcommand{\CommentTok}[1]{\textcolor[rgb]{0.56,0.35,0.01}{\textit{#1}}}
\newcommand{\CommentVarTok}[1]{\textcolor[rgb]{0.56,0.35,0.01}{\textbf{\textit{#1}}}}
\newcommand{\ConstantTok}[1]{\textcolor[rgb]{0.00,0.00,0.00}{#1}}
\newcommand{\ControlFlowTok}[1]{\textcolor[rgb]{0.13,0.29,0.53}{\textbf{#1}}}
\newcommand{\DataTypeTok}[1]{\textcolor[rgb]{0.13,0.29,0.53}{#1}}
\newcommand{\DecValTok}[1]{\textcolor[rgb]{0.00,0.00,0.81}{#1}}
\newcommand{\DocumentationTok}[1]{\textcolor[rgb]{0.56,0.35,0.01}{\textbf{\textit{#1}}}}
\newcommand{\ErrorTok}[1]{\textcolor[rgb]{0.64,0.00,0.00}{\textbf{#1}}}
\newcommand{\ExtensionTok}[1]{#1}
\newcommand{\FloatTok}[1]{\textcolor[rgb]{0.00,0.00,0.81}{#1}}
\newcommand{\FunctionTok}[1]{\textcolor[rgb]{0.00,0.00,0.00}{#1}}
\newcommand{\ImportTok}[1]{#1}
\newcommand{\InformationTok}[1]{\textcolor[rgb]{0.56,0.35,0.01}{\textbf{\textit{#1}}}}
\newcommand{\KeywordTok}[1]{\textcolor[rgb]{0.13,0.29,0.53}{\textbf{#1}}}
\newcommand{\NormalTok}[1]{#1}
\newcommand{\OperatorTok}[1]{\textcolor[rgb]{0.81,0.36,0.00}{\textbf{#1}}}
\newcommand{\OtherTok}[1]{\textcolor[rgb]{0.56,0.35,0.01}{#1}}
\newcommand{\PreprocessorTok}[1]{\textcolor[rgb]{0.56,0.35,0.01}{\textit{#1}}}
\newcommand{\RegionMarkerTok}[1]{#1}
\newcommand{\SpecialCharTok}[1]{\textcolor[rgb]{0.00,0.00,0.00}{#1}}
\newcommand{\SpecialStringTok}[1]{\textcolor[rgb]{0.31,0.60,0.02}{#1}}
\newcommand{\StringTok}[1]{\textcolor[rgb]{0.31,0.60,0.02}{#1}}
\newcommand{\VariableTok}[1]{\textcolor[rgb]{0.00,0.00,0.00}{#1}}
\newcommand{\VerbatimStringTok}[1]{\textcolor[rgb]{0.31,0.60,0.02}{#1}}
\newcommand{\WarningTok}[1]{\textcolor[rgb]{0.56,0.35,0.01}{\textbf{\textit{#1}}}}
\usepackage{graphicx,grffile}
\makeatletter
\def\maxwidth{\ifdim\Gin@nat@width>\linewidth\linewidth\else\Gin@nat@width\fi}
\def\maxheight{\ifdim\Gin@nat@height>\textheight\textheight\else\Gin@nat@height\fi}
\makeatother
% Scale images if necessary, so that they will not overflow the page
% margins by default, and it is still possible to overwrite the defaults
% using explicit options in \includegraphics[width, height, ...]{}
\setkeys{Gin}{width=\maxwidth,height=\maxheight,keepaspectratio}
\IfFileExists{parskip.sty}{%
\usepackage{parskip}
}{% else
\setlength{\parindent}{0pt}
\setlength{\parskip}{6pt plus 2pt minus 1pt}
}
\setlength{\emergencystretch}{3em}  % prevent overfull lines
\providecommand{\tightlist}{%
  \setlength{\itemsep}{0pt}\setlength{\parskip}{0pt}}
\setcounter{secnumdepth}{0}
% Redefines (sub)paragraphs to behave more like sections
\ifx\paragraph\undefined\else
\let\oldparagraph\paragraph
\renewcommand{\paragraph}[1]{\oldparagraph{#1}\mbox{}}
\fi
\ifx\subparagraph\undefined\else
\let\oldsubparagraph\subparagraph
\renewcommand{\subparagraph}[1]{\oldsubparagraph{#1}\mbox{}}
\fi

%%% Use protect on footnotes to avoid problems with footnotes in titles
\let\rmarkdownfootnote\footnote%
\def\footnote{\protect\rmarkdownfootnote}

%%% Change title format to be more compact
\usepackage{titling}

% Create subtitle command for use in maketitle
\providecommand{\subtitle}[1]{
  \posttitle{
    \begin{center}\large#1\end{center}
    }
}

\setlength{\droptitle}{-2em}

  \title{Charles Clarke}
    \pretitle{\vspace{\droptitle}\centering\huge}
  \posttitle{\par}
    \author{}
    \preauthor{}\postauthor{}
    \date{}
    \predate{}\postdate{}
  

\begin{document}
\maketitle

\hypertarget{r-markdown}{%
\subsection{R Markdown}\label{r-markdown}}

For this week's assignment we're going to use Dodgers Major League
Baseball data from 2012. The data file you will be using is contained in
the dodgers.csv file. I would like you to determine what night would be
the best to run a marketing promotion to increase attendance. It is up
to you if you decide to recommend a specific date (Jan 1, 2020) or if
you want to recommend a day of the week (Tuesdays) or Month and day of
the week (July Tuesdays). You will want to use TRAIN. As a reminder, the
training set is the data we fit our model on. Use a combination of R and
Python to accomplish this assignment. It is important to remember, there
will be lots of ways to solve this problem. Explain your thought process
and how you used various techniques to come up with your recommendation.
From this data, at a minimum, you should be able to demonstrate the
following:

Box plots

Scatter plots

Regression Model

\hypertarget{read-data-from-csv-file}{%
\subsubsection{Read Data from CSV file}\label{read-data-from-csv-file}}

\begin{Shaded}
\begin{Highlighting}[]
\NormalTok{MyData <-}\StringTok{ }\KeywordTok{read.csv}\NormalTok{(}\StringTok{"dodgers.csv"}\NormalTok{, }\DataTypeTok{header=}\OtherTok{TRUE}\NormalTok{, }\DataTypeTok{sep=}\StringTok{","}\NormalTok{)}
\end{Highlighting}
\end{Shaded}

\#Check for any NA data values

\begin{Shaded}
\begin{Highlighting}[]
\KeywordTok{anyNA}\NormalTok{(MyData)}
\end{Highlighting}
\end{Shaded}

\begin{verbatim}
## [1] FALSE
\end{verbatim}

\hypertarget{convert-string-variqbles-to-integer}{%
\subsection{Convert string variqbles to
integer}\label{convert-string-variqbles-to-integer}}

\begin{Shaded}
\begin{Highlighting}[]
\CommentTok{## Convert  string variqbles to integer}

\NormalTok{MyData}\OperatorTok{$}\NormalTok{attend <-}\StringTok{ }\KeywordTok{as.integer}\NormalTok{(MyData}\OperatorTok{$}\NormalTok{attend)}
\NormalTok{MyData}\OperatorTok{$}\NormalTok{temp <-}\StringTok{ }\KeywordTok{as.integer}\NormalTok{(MyData}\OperatorTok{$}\NormalTok{temp)}
\NormalTok{MyData}\OperatorTok{$}\NormalTok{day <-}\StringTok{ }\KeywordTok{as.integer}\NormalTok{(MyData}\OperatorTok{$}\NormalTok{day)}


\NormalTok{MyData}\OperatorTok{$}\NormalTok{day_of_week<-}\StringTok{ }\KeywordTok{as.factor}\NormalTok{(MyData}\OperatorTok{$}\NormalTok{day_of_week)}
\NormalTok{MyData}\OperatorTok{$}\NormalTok{day_night <-}\StringTok{ }\KeywordTok{as.factor}\NormalTok{(MyData}\OperatorTok{$}\NormalTok{day_night)}
\NormalTok{MyData}\OperatorTok{$}\NormalTok{skies <-}\StringTok{ }\KeywordTok{as.factor}\NormalTok{(MyData}\OperatorTok{$}\NormalTok{skies)}
\end{Highlighting}
\end{Shaded}

\hypertarget{dataframe-structure}{%
\subsection{Dataframe structure}\label{dataframe-structure}}

\begin{Shaded}
\begin{Highlighting}[]
\KeywordTok{str}\NormalTok{ (MyData)}
\end{Highlighting}
\end{Shaded}

\begin{verbatim}
## 'data.frame':    81 obs. of  12 variables:
##  $ month      : Factor w/ 7 levels "APR","AUG","JUL",..: 1 1 1 1 1 1 1 1 1 1 ...
##  $ day        : int  10 11 12 13 14 15 23 24 25 27 ...
##  $ attend     : int  56000 29729 28328 31601 46549 38359 26376 44014 26345 44807 ...
##  $ day_of_week: Factor w/ 7 levels "Friday","Monday",..: 6 7 5 1 3 4 2 6 7 1 ...
##  $ opponent   : Factor w/ 17 levels "Angels","Astros",..: 13 13 13 11 11 11 3 3 3 10 ...
##  $ temp       : int  67 58 57 54 57 65 60 63 64 66 ...
##  $ skies      : Factor w/ 2 levels "Clear ","Cloudy": 1 2 2 2 2 1 2 2 2 1 ...
##  $ day_night  : Factor w/ 2 levels "Day","Night": 1 2 2 2 2 1 2 2 2 2 ...
##  $ cap        : Factor w/ 2 levels "NO","YES": 1 1 1 1 1 1 1 1 1 1 ...
##  $ shirt      : Factor w/ 2 levels "NO","YES": 1 1 1 1 1 1 1 1 1 1 ...
##  $ fireworks  : Factor w/ 2 levels "NO","YES": 1 1 1 2 1 1 1 1 1 2 ...
##  $ bobblehead : Factor w/ 2 levels "NO","YES": 1 1 1 1 1 1 1 1 1 1 ...
\end{verbatim}

\hypertarget{print-summary}{%
\subsubsection{Print summary}\label{print-summary}}

\begin{verbatim}
##  month         day            attend         day_of_week      opponent 
##  APR:12   Min.   : 1.00   Min.   :24312   Friday   :13   Giants   : 9  
##  AUG:15   1st Qu.: 8.00   1st Qu.:34493   Monday   :12   Padres   : 9  
##  JUL:12   Median :15.00   Median :40284   Saturday :13   Rockies  : 9  
##  JUN: 9   Mean   :16.14   Mean   :41040   Sunday   :13   Snakes   : 9  
##  MAY:18   3rd Qu.:25.00   3rd Qu.:46588   Thursday : 5   Cardinals: 7  
##  OCT: 3   Max.   :31.00   Max.   :56000   Tuesday  :13   Brewers  : 4  
##  SEP:12                                   Wednesday:12   (Other)  :34  
##       temp          skies    day_night   cap     shirt    fireworks
##  Min.   :54.00   Clear :62   Day  :15   NO :79   NO :78   NO :67   
##  1st Qu.:67.00   Cloudy:19   Night:66   YES: 2   YES: 3   YES:14   
##  Median :73.00                                                     
##  Mean   :73.15                                                     
##  3rd Qu.:79.00                                                     
##  Max.   :95.00                                                     
##                                                                    
##  bobblehead
##  NO :70    
##  YES:11    
##            
##            
##            
##            
## 
\end{verbatim}

\begin{Shaded}
\begin{Highlighting}[]
\KeywordTok{library}\NormalTok{(corrplot)}
\end{Highlighting}
\end{Shaded}

\begin{verbatim}
## corrplot 0.84 loaded
\end{verbatim}

\begin{Shaded}
\begin{Highlighting}[]
\KeywordTok{anyNA}\NormalTok{(MyData)}
\end{Highlighting}
\end{Shaded}

\begin{verbatim}
## [1] FALSE
\end{verbatim}

\hypertarget{print-histogram-for-attendance.-distribution-of-attendance-and-frequencies.}{%
\subsubsection{Print Histogram for attendance. Distribution of
attendance and
frequencies.}\label{print-histogram-for-attendance.-distribution-of-attendance-and-frequencies.}}

\begin{Shaded}
\begin{Highlighting}[]
\KeywordTok{hist}\NormalTok{(MyData}\OperatorTok{$}\NormalTok{attend,}
\DataTypeTok{main =} \StringTok{"Histogram for Attendance"}\NormalTok{, }\DataTypeTok{xlab =} \StringTok{"Number at attendance"}\NormalTok{,}
\DataTypeTok{ylab =} \StringTok{"Frequency"}\NormalTok{)}
\end{Highlighting}
\end{Shaded}

\includegraphics{dodge_files/figure-latex/unnamed-chunk-6-1.pdf}

\hypertarget{look-at-correlation-between-attendance-and-temperature}{%
\subsubsection{Look at correlation between attendance and
temperature}\label{look-at-correlation-between-attendance-and-temperature}}

\begin{Shaded}
\begin{Highlighting}[]
 \KeywordTok{cor}\NormalTok{(MyData}\OperatorTok{$}\NormalTok{attend,MyData}\OperatorTok{$}\NormalTok{temp)}
\end{Highlighting}
\end{Shaded}

\begin{verbatim}
## [1] 0.09895073
\end{verbatim}

\hypertarget{plot-scatter.-attendance-vs-temperature.}{%
\subsection{Plot scatter. Attendance vs
Temperature.}\label{plot-scatter.-attendance-vs-temperature.}}

\begin{Shaded}
\begin{Highlighting}[]
\KeywordTok{ggplot}\NormalTok{(MyData, }\KeywordTok{aes}\NormalTok{(}\DataTypeTok{y =}\NormalTok{ MyData}\OperatorTok{$}\NormalTok{attend, }\DataTypeTok{x=}\NormalTok{ MyData}\OperatorTok{$}\NormalTok{temp)) }\OperatorTok{+}\StringTok{ }\KeywordTok{xlab}\NormalTok{(}\StringTok{"Temperture"}\NormalTok{) }\OperatorTok{+}\StringTok{ }\KeywordTok{ylab}\NormalTok{(}\StringTok{"Attendance"}\NormalTok{)}\OperatorTok{+}
\StringTok{         }\KeywordTok{geom_point}\NormalTok{(}\DataTypeTok{size=}\DecValTok{2}\NormalTok{, }\DataTypeTok{shape=}\DecValTok{23}\NormalTok{)}
\end{Highlighting}
\end{Shaded}

\includegraphics{dodge_files/figure-latex/unnamed-chunk-8-1.pdf}

\begin{Shaded}
\begin{Highlighting}[]
  \CommentTok{#geom_bar( stat = "identity") }
\end{Highlighting}
\end{Shaded}

\#\#\#Multivariate Plots

\begin{Shaded}
\begin{Highlighting}[]
\KeywordTok{featurePlot}\NormalTok{(}\DataTypeTok{x=}\NormalTok{MyData}\OperatorTok{$}\NormalTok{attend, }\DataTypeTok{y=}\NormalTok{MyData}\OperatorTok{$}\NormalTok{temp, }\DataTypeTok{plot=}\StringTok{"ellipse"}\NormalTok{)}
\end{Highlighting}
\end{Shaded}

\begin{verbatim}
## NULL
\end{verbatim}

\hypertarget{box-plots-of-attendance-vs-day-of-week}{%
\subsubsection{Box plots of Attendance vs Day Of
Week}\label{box-plots-of-attendance-vs-day-of-week}}

Monday has two outlieners. Monday showing one day with hight attendance
and low attendance occurances.

Tuesday has the highest attendance number. Attendance number above the
mean value.

\begin{Shaded}
\begin{Highlighting}[]
\KeywordTok{plot}\NormalTok{(MyData}\OperatorTok{$}\NormalTok{day_of_week, MyData}\OperatorTok{$}\NormalTok{attend,}
\DataTypeTok{main =} \StringTok{"Attendance vs Day of Week"}\NormalTok{,}
\DataTypeTok{ylab =} \StringTok{"Attendance number"}\NormalTok{,}
\DataTypeTok{xlab =} \StringTok{"Day of Week"}\NormalTok{)}
\end{Highlighting}
\end{Shaded}

\includegraphics{dodge_files/figure-latex/unnamed-chunk-10-1.pdf} \#\#\#
split data for training and testing

\begin{Shaded}
\begin{Highlighting}[]
\NormalTok{samp_size =}\StringTok{ }\KeywordTok{createDataPartition}\NormalTok{(MyData}\OperatorTok{$}\NormalTok{attend,}\DataTypeTok{p=}\NormalTok{.}\DecValTok{75}\NormalTok{,}\DataTypeTok{list =} \OtherTok{FALSE}\NormalTok{)}
\CommentTok{#smp_size <- floor(0.75 * nrow(MyData))}



\NormalTok{training <-MyData[samp_size, ]}
\NormalTok{testing <-}\StringTok{ }\NormalTok{MyData[}\OperatorTok{-}\NormalTok{samp_size, ]}
\end{Highlighting}
\end{Shaded}

\#Print the training and testing dataset after the split

\begin{Shaded}
\begin{Highlighting}[]
\KeywordTok{print}\NormalTok{(}\StringTok{"complete dataset -->"}\NormalTok{)}
\end{Highlighting}
\end{Shaded}

\begin{verbatim}
## [1] "complete dataset -->"
\end{verbatim}

\begin{Shaded}
\begin{Highlighting}[]
\KeywordTok{dim}\NormalTok{(MyData)}
\end{Highlighting}
\end{Shaded}

\begin{verbatim}
## [1] 81 12
\end{verbatim}

\begin{Shaded}
\begin{Highlighting}[]
\KeywordTok{print}\NormalTok{ (}\StringTok{"Train dataset ---> "}\NormalTok{)}
\end{Highlighting}
\end{Shaded}

\begin{verbatim}
## [1] "Train dataset ---> "
\end{verbatim}

\begin{Shaded}
\begin{Highlighting}[]
\KeywordTok{dim}\NormalTok{(training)}
\end{Highlighting}
\end{Shaded}

\begin{verbatim}
## [1] 61 12
\end{verbatim}

\begin{Shaded}
\begin{Highlighting}[]
\KeywordTok{print}\NormalTok{ (}\StringTok{"test dataset ---> "}\NormalTok{)}
\end{Highlighting}
\end{Shaded}

\begin{verbatim}
## [1] "test dataset ---> "
\end{verbatim}

\begin{Shaded}
\begin{Highlighting}[]
\KeywordTok{dim}\NormalTok{(testing)}
\end{Highlighting}
\end{Shaded}

\begin{verbatim}
## [1] 20 12
\end{verbatim}

\hypertarget{run-linear-regression-model-and-print-summary}{%
\subsubsection{Run linear regression model and print
summary}\label{run-linear-regression-model-and-print-summary}}

\begin{Shaded}
\begin{Highlighting}[]
\NormalTok{linear_model<-}\KeywordTok{train}\NormalTok{(attend }\OperatorTok{~}\StringTok{ }\KeywordTok{factor}\NormalTok{(day_of_week)}\OperatorTok{+}\KeywordTok{factor}\NormalTok{(day_night), }\DataTypeTok{data =}\NormalTok{ training)}

\KeywordTok{print}\NormalTok{(}\KeywordTok{summary}\NormalTok{(linear_model))}
\end{Highlighting}
\end{Shaded}

\begin{verbatim}
##                 Length Class      Mode     
## call              4    -none-     call     
## type              1    -none-     character
## predicted        61    -none-     numeric  
## mse             500    -none-     numeric  
## rsq             500    -none-     numeric  
## oob.times        61    -none-     numeric  
## importance        7    -none-     numeric  
## importanceSD      0    -none-     NULL     
## localImportance   0    -none-     NULL     
## proximity         0    -none-     NULL     
## ntree             1    -none-     numeric  
## mtry              1    -none-     numeric  
## forest           11    -none-     list     
## coefs             0    -none-     NULL     
## y                61    -none-     numeric  
## test              0    -none-     NULL     
## inbag             0    -none-     NULL     
## xNames            7    -none-     character
## problemType       1    -none-     character
## tuneValue         1    data.frame list     
## obsLevels         1    -none-     logical  
## param             0    -none-     list
\end{verbatim}

\hypertarget{linear-regression.-print-model-values}{%
\subsubsection{Linear Regression. Print model
values}\label{linear-regression.-print-model-values}}

\begin{Shaded}
\begin{Highlighting}[]
\KeywordTok{print}\NormalTok{(linear_model )  }\CommentTok{# }
\end{Highlighting}
\end{Shaded}

\begin{verbatim}
## Random Forest 
## 
## 61 samples
##  2 predictor
## 
## No pre-processing
## Resampling: Bootstrapped (25 reps) 
## Summary of sample sizes: 61, 61, 61, 61, 61, 61, ... 
## Resampling results across tuning parameters:
## 
##   mtry  RMSE      Rsquared    MAE     
##   2     8310.274  0.09004252  6753.851
##   4     8575.526  0.08437324  6946.705
##   7     8657.820  0.08283480  7026.828
## 
## RMSE was used to select the optimal model using the smallest value.
## The final value used for the model was mtry = 2.
\end{verbatim}


\end{document}
